%% Copernicus Publications Manuscript Preparation Template for LaTeX Submissions
%% ---------------------------------
%% This template should be used for copernicus.cls
%% The class file and some style files are bundled in the Copernicus Latex Package, which can be downloaded from the different journal webpages.
%% For further assistance please contact Copernicus Publications at: production@copernicus.org
%% https://publications.copernicus.org/for_authors/manuscript_preparation.html


%% Please use the following documentclass and journal abbreviations for preprints and final revised papers.

%% 2-column papers and preprints
\documentclass[journal abbreviation, manuscript]{copernicus}



%% Journal abbreviations (please use the same for preprints and final revised papers)


% Advances in Geosciences (adgeo)
% Advances in Radio Science (ars)
% Advances in Science and Research (asr)
% Advances in Statistical Climatology, Meteorology and Oceanography (ascmo)
% Aerosol Research (ar)
% Annales Geophysicae (angeo)
% Archives Animal Breeding (aab)
% Atmospheric Chemistry and Physics (acp)
% Atmospheric Measurement Techniques (amt)
% Biogeosciences (bg)
% Climate of the Past (cp)
% DEUQUA Special Publications (deuquasp)
% Earth Surface Dynamics (esurf)
% Earth System Dynamics (esd)
% Earth System Science Data (essd)
% E&G Quaternary Science Journal (egqsj)
% EGUsphere (egusphere) | This is only for EGUsphere preprints submitted without relation to an EGU journal.
% European Journal of Mineralogy (ejm)
% Fossil Record (fr)
% Geochronology (gchron)
% Geographica Helvetica (gh)
% Geoscience Communication (gc)
% Geoscientific Instrumentation, Methods and Data Systems (gi)
% Geoscientific Model Development (gmd)
% History of Geo- and Space Sciences (hgss)
% Hydrology and Earth System Sciences (hess)
% Journal of Bone and Joint Infection (jbji)
% Journal of Micropalaeontology (jm)
% Journal of Sensors and Sensor Systems (jsss)
% Magnetic Resonance (mr)
% Mechanical Sciences (ms)
% Natural Hazards and Earth System Sciences (nhess)
% Nonlinear Processes in Geophysics (npg)
% Ocean Science (os)
% Polarforschung - Journal of the German Society for Polar Research (polf)
% Primate Biology (pb)
% Proceedings of the International Association of Hydrological Sciences (piahs)
% Safety of Nuclear Waste Disposal (sand)
% Scientific Drilling (sd)
% SOIL (soil)
% Solid Earth (se)
% State of the Planet (sp)
% The Cryosphere (tc)
% Weather and Climate Dynamics (wcd)
% Web Ecology (we)
% Wind Energy Science (wes)


%% \usepackage commands included in the copernicus.cls:
%\usepackage[german, english]{babel}
%\usepackage{tabularx}
%\usepackage{cancel}
%\usepackage{multirow}
%\usepackage{supertabular}
%\usepackage{algorithmic}
%\usepackage{algorithm}
%\usepackage{amsthm}
%\usepackage{float}
%\usepackage{subfig}
%\usepackage{rotating}

\lstset{language=[90]Fortran,
  basicstyle=\ttfamily,
  keywordstyle=\color{red},
  commentstyle=\color{green},
  morecomment=[l]{!\ }% Comment only with space after !
}

\begin{document}

\title{RegCM-ES v1.1 Regional Earth System Model: model components update and
performance evaluation over the MED-cordex region}


% \Author[affil]{given_name}{surname}

\Author[1][ggiulian@ictp.it]{Graziano}{Giuliani} %% correspondence author
\Author[2][]{Marco}{Reale}
\Author[1][]{Tapajyoti}{Chakraborty}
\Author[1][]{Erika}{Coppola}
\Author[2][]{Stefano}{Salon}

\affil[1]{Earth System Physics Section, The Abdus Salam International Centre for Theoretical Physics, Trieste, Italy}
\affil[2]{National Institute of Oceanography and Applied Geophysics—OGS, Triete, Italy}

%% The [] brackets identify the author with the corresponding affiliation. 1, 2, 3, etc. should be inserted.


%% If authors contributed equally, please add \equalcontrib{$Author numbers$} (e.g. \equalcontrib{1,3}) at the end of the affiliations. The author number depends on the placement of the author in the author list, e.g. the third author has number 3.


\runningtitle{RegCM-ES1-1 Regional Earth System Model}

\runningauthor{Graziano Giuliani}

\received{}
\pubdiscuss{} %% only important for two-stage journals
\revised{}
\accepted{}
\published{}

%% These dates will be inserted by Copernicus Publications during the typesetting process.


\firstpage{1}

\maketitle



\begin{abstract}
The need to understand the long term evolution and complex interactions of
    the Earth's climate system physical components, Atmosphere, Ocean, Land
    and Cryosphere, and investigate the impacts of the human generated
    modifications to the energy equilibrium state and the planet biosphere,
    is fulfilled with interacting models simulations orchestrated in coupled
    Earth System Models (ESM).
    The necessity is still compelling at regional scales, in the areas where
    the specific climate is known to be driven by the interaction of the
    different subsystems, like the Mediterranenan basin.
    In the framework of the MED-Cordex experiment, we have seen the
    implementation of Regional Coupled Atmosphere, Ocean and River
    Climate Models (AORCM) to obtain a more accurate representation of the
    Earth System.
    One of the model used is the ICTP RegCM-ES.
    We will describe in this paper the recent changes to the different model
    components implemented jointly by the Abdus Salam International Centre
    of Theoretical Physics (ICTP) and the Italian National Institute of
    Oceanography and Applied Geophysics (INOGS).
    The aim is to describe the effort in updating the codebase of the
    three models, with a focus on the resulting better integration of the
    different parts into the RegESM coupler, and the new baseline
    performances of the updated model over the whole Mediterranean
    and Black Sea region.
\end{abstract}


\copyrightstatement{This work is distributed under the Creative Commons Attribution 4.0 License.} %% This section is optional and can be used for copyright transfers.


\introduction  %% \introduction[modified heading if necessary]
A coupled Earth System Model (ESM) is not just a computer program, but a
successful interaction of multiple scientific communities agreeing to
participate in a joint effort with the goal to describe the physical
state and behavior of the most complex system scientists can put their hands
on: our planet.

Throughout the last decades, broad communities have jointly participated in
the writing of multiple flexible and high performance software libraries
which have permitted the building of ESMs from interoperable components.

One of these software libraries is the Earth System Modeling Framework ESMF,
\citet{esmf}.
It has been developed by the United States University Corporation for
Atmospheric Research (UCAR) through a joint specification team of
researchers from various institutions, including the National Center for
Atmospheric Research (NCAR), the National Oceanic and Atmospheric
Administration (NOAA), the Geophysical Fluid Dynamics Laboratory (GFDL).

The platform functionalities have been successfully exploited through the
ESMF National Unified Operational Prediction Capability (NUOPC)
interoperability layer to create the model coupler RegESM,
\citep{ufuk-gmd-2013,ufuk-cd-2017}, to provide an efficient framework for
the integration of single component models into a Regional Earth System Model.

Following the common and well described conventions, the RegESM permits the
integration of multiple components by defining the synchronization and data
exchange of their interactions, to reflect the different timescales of the
phisical processes they describe. It manages all the required interpolations
and unit of measure harmonization needed for the flow of mass, momentum and
energy fluxes between the component models.

With minor modifications required into the single model codebases, the
RegESM has created a complete integration framework for the creation of
"a la carte" composition of multiple pluggable models for
the different Atmosphere, Ocean, River and Wave components, with the
possibility to use high performance visualization tools
to analyze on-the-fly the produced results \citep{ufuk-gmd-2019}.

Different research groups are successfully using the RegESM framework to
create coupled regional Earth System Models: in Italy the ENEA with their
ENEA-Reg \citep{anav-gmd-2021} coupling the Atmosphere and Land model
WRF \citep{wrf-2017} with the MITgcm Ocean model
\citep{marshall-1997,adcroft-2004} and the HD river \citep{hagemann-1997},
in Turkey the ITU \citep{batibeniz-2025} which have implemented
the coupling with the WAM \citep{wamdi-1988} wave model.

In ICTP and OGS, this capability has been successfully exploited in past years
by creating the RegCM-ES model, composed by coupling together, with the help
of the RegESM, the RegCM 4 atmospheric model with the CLM4.5 land model
\citep{giorgi-2012,oleson-2013}, the MITgcm Ocean model with
the eventual Biogeophisics component \citep{cossarini-gmd-2017},
and the HD river model.

You can find in \citet{sitz-2017} the first description of the ICTP RegCM-ES
model, and in \citet{reale-2020} the description of the model performances over
the Mediterranean region when coupled with the Biogeochemical Marine flux
model for the study of the marine ecosystem.

We describe in this paper the updated ICTP RegCM-ES model, which is now
composed by:

\begin{enumerate}
    \item The Atmospheric model RegCM version 5 \citep{giorgi-2023}
        internally coupled with the CLM4.5 land model \citep{oleson-2013}
    \item The Ocean model MITgcm version 69 \citep{marshall-1997,adcroft-2004}
    \item The CETEMPS River routing CHyM with runoff input, implemented
        for the ESM coupling \citep{coppola-2007}
\end{enumerate}

The so composed ICTP AORCM Regional Earth System Model will be referred in the
following as the RegCM-ES-1-1 and will be used by the ICTP in
the Coordinated Regional Climate Downscaling Experiment (CORDEX) for the
Phase 2 Mediterranean domain Med-CORDEX experiment \citep{somot-2020}.

\section{Coupled model compoenents}

We give here a description of the RegCM-ES-1-1 model components used in the
evaluation of the model performances, describing eventual changes from the
those presented in previous literature and the limits of the coupling
system.

\subsection{Coupler}

The RegESM coupled modeling system is based on the ESMF framework and is
mostly following the design described in previous papers. The major
modifications in the code are:

\begin{enumerate}
    \item The update to use the more recent ESMF library version 8.8.1
    \item The addition of the interface with a special version of the CHyM
        model into the list of the supported River Routing models.
    \item The update to support the most recent version of the RegCM model
    \item The update to support the most recent version of the MITgcm model
\end{enumerate}

The updated code is available on GitHub (see below in the section \ref{code}).

The RegESM framework creates a single model executable file by
linking together Fortran module objects from all the component models, each
wrapped as a static library and selected at the configure step by the user
by specifying the paths to the already compiled and coupling configured
model components. A simple check of the source files is performed to recognize
at this stage which model component has been selected among the possible
options for the atmosphere, ocean and river, and the relevant interface
modules are compiled. The documentation available in the GitHub page and
the scripts made available should guide any user in compiling and setting
up a successful coupled model simulation. Due to a possible misconfiguration
of the target system used in the simulation run, we have been limited to use
the sequential execution scheme in the experiment below described.

The set of variables (described in Appendix \ref{appendix:regesm}) is chosen to
keep the radiation balance internal to the atmospheric model by providing to
the ocean the surface net short and long wave components.

We allow for an independent computation in both the atmosphere and
the ocean model of the bulk energy and mass transfer, by feeding to the ocean
the input wind zonal and meridional components already scaled to anemometric
height, togheter with the air temperature and specific humidity at $2m$ height.

The set of variables from the atmosphere to the ocean is completed with the
precipitation and surface air pressure.

The ocean reaceives these fields with hourly time resolution (following the
MED-Cordex protocol). The river discharge from the routing model is updated
instead with daily frequency to the ocean model. The atmosphere provides
to the CHyM as input the daily average total runoff. The exchange feedback
is closed with the hourly sst from the ocean to the atmosphere.
The timing of the interactions is specified in the input file for the RegESM
as reported in the excerpt in listing Appendix \ref{appendix:regesm}.

\subsection{Atmosphere Model}

The RegCM model is widely considered the first Regional Climate model
documented in the literature \citep{giorgi-1989}, and has seen throughout
the years constant development efforts
\citep{giorgi-1993,giorgi-1999,pal-2007,giorgi-2012,coppola-2021,giorgi-2023},
leading to successive model releases with the latest revision being the
RegCM version 5 model.

The Atmospheric model supports a number of different dynamical and physical
packages \citep{giorgi-2023}, which create a flexible tool used by multiple
research groups all over the world for the study of regional climate by
running long term simulations in the framework of the World Climate
Research Program (WCRP) CORDEX experiment, process analyses to understand
the regional climate, or as the atmospheric component in different
coupled atmosphere and ocean studies
\citep{perrie-2001,ufuk-gmd-2013,wei-2014,ratnam-2009,sitz-2017,
       disante-2019,reale-2020,mishra-2021}.

The model has hydrostatic \citep{anthes-1987} and two non-hydrostatic dynamical
descriptions (MM5, \citet{grell-1994} and MOLOCH, \citet{davolio-2020}) of
the basic time evolution of the atmosphere status. The surface status is
described by using different land models (BATS \citet{dickinson-1993}
and CLM4.5 \citet{oleson-2013}).
The boundary layer cascade energy transfer is possible with local and
non-local schemes \citep{holtslag-1989,park-2009}.
The parametrized convection is available using multiple possible schemes
\citep{emanuel-1991,tiedtke-1989,kain-2004}, while the microphysical
description is possible using different packages
\citep{pal-2000,lim-2005,noto-gmd-2016}.
The cloud fraction scheme can be chosen among the \citet{sundqvist-1989}
and the \citet{xu-1996}.
The bulk parameterization of the Air-Sea fluxes uses either the COARE
\citep{fairall-2003} algorithm or the \citet{zeng-2005} scheme.
Two different radiation transmission schemes are available, one from the
CCM3 \citep{kiehl-1996}, the other being the ubiquitous RRTMG
\citep{mlawer-1997,iacono-2008}.
An optional aerosol \citep{zakey-2006,zakey-2008,solmon-2008} and
full chemistry package \citep{shalaby-2012} is available.
The model has full integration with Input4MIPS \citep{durack-2017}
dataset to be used to downscale to the regional scale the Global
Climate Models (GCMs) participating in the CMIP5 \citep{taylor-2012}
and CMIP6 \citep{eyring-2016} experiments.

Input packages permit the interpolation as lateral boundary conditions
at configurable time frequencies of idifferent reanalysis dataset, ERA5
\citep{hershbach-2020} among them, and multiple GCM global models.
The coupling of the model to be part of an AORCM is possible using both
the ESMF through the RegESM and the OASIS \citep{craig-2017} frameworks.
The model has been successfully used to obtain future regiobal climate
projections in the framework of the CORDEX \citep{gutowski-2016}
and CORDEX-CORE (evaluated in \citet{remedio-2019}) experiments.

The model has a large user community spanning at least six continents, and its
development is currently coordinated by the UNESCO ICTP in Trieste.

In the framework of past CORDEX experiment, the model has been used to simulate
future climate over most of the experiment defined regional domains, with the
exception of the high latitude Arctic and Antartic regions.

\begin{table}[]
    \begin{tabular}{|l|l|}
        \hline
        Package & Scheme \\
        \hline
        Dynamic & MOLOCH non hydrostatic \\
        Land & CLM4.5 \\
        Radiation & CCSM3 \\
        Microphysics & Nogherotto-Tompkins \\
        Convection & Tiedtke \\
        Planet BL & Holtslag \\
        Clouds & Sundqvist \\
        Ocean Bulk flux & Zeng \\
        \hline
  \end{tabular}
    \caption{RegCM model configuration}
    \label{table:regcm_conf}
\end{table}

The Med-Cordex experiment settings here reported in the description of the
RegCM-ES-1-1 are in table \ref{table:regcm_conf}.

The RegCM5, in the framework of the Med-CORDEX experiment, is using a partial
set of Input4MIPS data (GHG concentrations, O3, Solar radiation). For the
evaluation in this paper, the model is using a reanalysis dataset of Aerosol
optical properties from the Modern-Era Retrospective analysis for Research
and Applications, Version 2 (MERRA2) \citep{randles-2017,buchard-2017},
while the CLM4.5 input dataset is not changing throughout the simulation
timeframe.

\subsection{Ocean}

The MITgcm (MIT General Circulation Model) is a numerical model designed for
study of the atmosphere, ocean, and climate. The Ocean component has been
coupled to the RegCM using the


\subsection{River}

The ICTP CHyM model used in the RegCM-ES1-1 is a trimmed down version of the
CETEMPS CHyM model where the soil model accounting for infiltration is turned
off and only the water transmission through the momentum and continuity
equations for the surface free flow are considered. We present in the
Appendix \ref{appendix:chym} the model equations because the CHyM is the new
component for the RegCM-ES1-1.

Input to the model is the runoff physical variable, which is a common product
of any Land model component of most NWP or GCM models, and is generally defined
as the portion of precipitation that doesn't infiltrate into the soil or
evaporate and flow towards streams, rivers, and other bodies of water.

The river discharge at the river mouths in output is then provided as the
freswater input to the MITgcm model by interpolating it to the nearest MITgcm
model gridcell and considering a diffusion radius of $25 km$
(hardcoded in the REGESM).

The calibration of the coupled CHyM model for the above parameters has been
done using the CHyM-roff model and the ECMWF ERA5 total runoff product as
input runoff. The results of this calibration are presented by providing the
point plots of the river discharge at selected measurement stations at the
river mouths available in the RivDIS dataset obtained using the
settings in table \ref{table:chym_conf}.

\begin{table}[]
    \begin{tabular}{|l|l|}
        \hline
        Parameter & Value \\
        \hline
        $\alpha$ & $0.0015$ \\
        $\beta$ & $0.050$ \\
        $\gamma$ & $0.33$ \\
        $\delta$ & $4.5$ \\
        $D_m$ & $100$ \\
        \hline
  \end{tabular}
    \caption{CHyM model configuration}
    \label{table:chym_conf}
\end{table}

The Manning's roughness coefficients are liberally deived from a table
present in the US Army Corp of Engineers River Analysis Sistem (HEC-RAS)
Reference Manual \citep{brunner-2020} and are described in Appendix
\ref{appendix:chym}.

The slope and river hydraulic radio are estimated using the Hydrosheds
[data: \ref{data:hydrosheds}], while the surface gridcell classification is
derived from GLCC [data: \ref{data:glcc}].

\conclusions  %% \conclusions[modified heading if necessary]

Magari ci arriviamo pure...

%% The following commands are for the statements about the availability of data sets and/or software code corresponding to the manuscript.
%% It is strongly recommended to make use of these sections in case data sets and/or software code have been part of your research the article is based on.

\codeavailability{
\label{code}

The different models' code used in the described RegCM-ES-1-1 are all available
on GitHub under different type of License. The scientific software libraries
used by the different model components are not here listed and reported: it
is left to the user to look into the ESMF library documentation for the list
of all required and optional dependencies.

\begin{enumerate}
    \item RegESM model framework. The code has been forked from the reference
        repository and the updated code is available at the following URL:
        \url{https://github.com/graziano-giuliani/RegESM}. It is distributed
        under the MIT license.
    \item RegCM atmosphere and land model component. The code used in the
        experiment is from a branch of the ICTP repository used for the
        CORDEX experiment, and is available at the following URL:
        \url{https://github.com/ICTP/RegCM/tree/CORDEX-5}. The interface is
        compatible with the latest model version, distributed under the MIT
        license.
    \item MITgcm ocean model component. The code used in the experiment is
        available from the MITgcm GitHub page at the following URL:
        \url{https://github.com/MITgcm/MITgcm}. The data preprocessing
        scripts and the copling code modifications are available in a
        dedicated repository at the following URL:
        \url{https://github.com/graziano-giuliani/MED12-ocean-mit}.
        The model us distributed under the MIT license.
    \item CHyM river routing in couplig configuration is available at the
        following URL: \url{https://github.com/graziano-giuliani/CHyM\_cpl}.
        The model static data creation is performed using the uncoupled
        model version code available at the following URL:
        \url{https://github.com/graziano-giuliani/CHyM-roff}. Both models
        are distributed under the GPL v3 license.
    \item All the scripts used for plotting and rendering of the results
        presented in this paper are available at the following GitHub URL:
        \url{https://github.com/graziano-giuliani/MED12-ocean-mit}.
\end{enumerate}
}

\dataavailability{
    Here following the list of datasets used for this work:
\begin{enumerate}
    \item \label{data:hydrosheds} Hydrosheds :
        \citep{lehner-2008}, \cite{hydrosheds-1.1}
    \item \label{data:glcc} GLCC : \citep{glcc-2.0}
    \item \label{data:rivdis} RivDIS : \citep{rivdis-1.1}
\end{enumerate}
}

\sampleavailability{TEXT} %% use this section when having geoscientific samples available


\videosupplement{TEXT} %% use this section when having video supplements available


\appendix
\section{RegESM configuration}
\label{appendix:regesm}
In listing \ref{lst:exfields} the reader can find the defined exchange
set of the variables between the models.

\lstinputlisting[caption=Exchange Table, label={lst:exfields}, language=C]{exfield.tbl}

In listing \ref{lst:namelistrc} the reader can find the timing of the
exchanges between the different models.

\lstinputlisting[caption=Exchange Timing, label={lst:namelistrc}, language=bash]{namelist.rc}

\section{CHyM model}
\label{appendix:chym}

The Chym model uses Manning's empirical momentum equation \citep{manning-1891}
for the flow rate of water discharge $V$ [Eq. \ref{eqn:chym_equations}]
considering the slope $S$ and the hydraulic radius $R$:

\begin{align}
    \label{eqn:chym_equations}
    V &= \frac{\sqrt{S} R^{\frac{2}{3}}}{n(M)} \\
    R &= \alpha + \beta \max(D,D_{m})^{\gamma} \\
    n(M) &= \begin{cases}
      \frac{M}{\delta}, & \text{if } D > D_{m} \\
      \frac{M}{1+(\delta-1) \frac{1+(D-D_{m})}{D_{m}}}, & \text{if } D < D_{m}
    \end{cases}
\end{align}

The hydraulic radius is linearly dependent on the drained area $D$, which
is computed using the Cellular Automata algorithm described in
\cite{coppola-2007}, and $\alpha$, $\beta$, $\gamma$, $\delta$ and $D_{m}$ are
calibration coefficients.
$M$ are the land use dependent Manning's roughness coefficients in
$(s m^{-1})^{\frac{1}{3}}$.

\lstinputlisting[caption=Manning coefficients, label={lst:chym}, language=Fortran]{manning}

The model uses the continuity equation for the total water mass $A$ per unit
lenght [Eq. \ref{eqn:chym_water_continuity}] with and the input runoff per unit
lenght $q_c$, to compute the river discharge $P$ in $m^3 s^{-1}$ by time
integration with a suitable time step.

\begin{align}
    \label{eqn:chym_water_continuity}
    \frac{\partial A}{\partial t} + \frac{\partial V}{\partial x} &= q_c \\
    P &= V A
\end{align}

\noappendix       %% use this to mark the end of the appendix section. Otherwise the figures might be numbered incorrectly (e.g. 10 instead of 1).

%% Regarding figures and tables in appendices, the following two options are possible depending on your general handling of figures and tables in the manuscript environment:

%% Option 1: If you sorted all figures and tables into the sections of the text, please also sort the appendix figures and appendix tables into the respective appendix sections.
%% They will be correctly named automatically.

%% Option 2: If you put all figures after the reference list, please insert appendix tables and figures after the normal tables and figures.
%% To rename them correctly to A1, A2, etc., please add the following commands in front of them:

\appendixfigures  %% needs to be added in front of appendix figures

\appendixtables   %% needs to be added in front of appendix tables

%% Please add \clearpage between each table and/or figure. Further guidelines on figures and tables can be found below.



\authorcontribution{TEXT} %% this section is mandatory

\competinginterests{TEXT} %% this section is mandatory even if you declare that no competing interests are present

\disclaimer{TEXT} %% optional section

\begin{acknowledgements}
TEXT
\end{acknowledgements}




%% REFERENCES

%% The reference list is compiled as follows:

%\begin{thebibliography}{}
%\end{thebibliography}

%% Since the Copernicus LaTeX package includes the BibTeX style file copernicus.bst,
%% authors experienced with BibTeX only have to include the following two lines:
%%
\bibliographystyle{copernicus}
\bibliography{bibliography.bib}
%%
%% URLs and DOIs can be entered in your BibTeX file as:
%%
%% URL = {http://www.xyz.org/~jones/idx_g.htm}
%% DOI = {10.5194/xyz}


%% LITERATURE CITATIONS
%%
%% command                        & example result
%% \citet{jones90}|               & Jones et al. (1990)
%% \citep{jones90}|               & (Jones et al., 1990)
%% \citep{jones90,jones93}|       & (Jones et al., 1990, 1993)
%% \citep[p.~32]{jones90}|        & (Jones et al., 1990, p.~32)
%% \citep[e.g.,][]{jones90}|      & (e.g., Jones et al., 1990)
%% \citep[e.g.,][p.~32]{jones90}| & (e.g., Jones et al., 1990, p.~32)
%% \citeauthor{jones90}|          & Jones et al.
%% \citeyear{jones90}|            & 1990



%% FIGURES

%% When figures and tables are placed at the end of the MS (article in one-column style), please add \clearpage
%% between bibliography and first table and/or figure as well as between each table and/or figure.

% The figure files should be labelled correctly with Arabic numerals (e.g. fig01.jpg, fig02.png).


%% ONE-COLUMN FIGURES

%%f
%\begin{figure}[t]
%\includegraphics[width=8.3cm]{FILE NAME}
%\caption{TEXT}
%\end{figure}
%
%%% TWO-COLUMN FIGURES
%
%%f
%\begin{figure*}[t]
%\includegraphics[width=12cm]{FILE NAME}
%\caption{TEXT}
%\end{figure*}
%
%
%%% TABLES
%%%
%%% The different columns must be seperated with a & command and should
%%% end with \\ to identify the column brake.
%
%%% ONE-COLUMN TABLE
%
%%t
%\begin{table}[t]
%\caption{TEXT}
%\begin{tabular}{column = lcr}
%\tophline
%
%\middlehline
%
%\bottomhline
%\end{tabular}
%\belowtable{} % Table Footnotes
%\end{table}
%
%%% TWO-COLUMN TABLE
%
%%t
%\begin{table*}[t]
%\caption{TEXT}
%\begin{tabular}{column = lcr}
%\tophline
%
%\middlehline
%
%\bottomhline
%\end{tabular}
%\belowtable{} % Table Footnotes
%\end{table*}
%
%%% LANDSCAPE TABLE
%
%%t
%\begin{sidewaystable*}[t]
%\caption{TEXT}
%\begin{tabular}{column = lcr}
%\tophline
%
%\middlehline
%
%\bottomhline
%\end{tabular}
%\belowtable{} % Table Footnotes
%\end{sidewaystable*}
%
%
%%% MATHEMATICAL EXPRESSIONS
%
%%% All papers typeset by Copernicus Publications follow the math typesetting regulations
%%% given by the IUPAC Green Book (IUPAC: Quantities, Units and Symbols in Physical Chemistry,
%%% 2nd Edn., Blackwell Science, available at: http://old.iupac.org/publications/books/gbook/green_book_2ed.pdf, 1993).
%%%
%%% Physical quantities/variables are typeset in italic font (t for time, T for Temperature)
%%% Indices which are not defined are typeset in italic font (x, y, z, a, b, c)
%%% Items/objects which are defined are typeset in roman font (Car A, Car B)
%%% Descriptions/specifications which are defined by itself are typeset in roman font (abs, rel, ref, tot, net, ice)
%%% Abbreviations from 2 letters are typeset in roman font (RH, LAI)
%%% Vectors are identified in bold italic font using \vec{x}
%%% Matrices are identified in bold roman font
%%% Multiplication signs are typeset using the LaTeX commands \times (for vector products, grids, and exponential notations) or \cdot
%%% The character * should not be applied as mutliplication sign
%
%
%%% EQUATIONS
%
%%% Single-row equation
%
%\begin{equation}
%
%\end{equation}
%
%%% Multiline equation
%
%\begin{align}
%& 3 + 5 = 8\\
%& 3 + 5 = 8\\
%& 3 + 5 = 8
%\end{align}
%
%
%%% MATRICES
%
%\begin{matrix}
%x & y & z\\
%x & y & z\\
%x & y & z\\
%\end{matrix}
%
%
%%% ALGORITHM
%
%\begin{algorithm}
%\caption{...}
%\label{a1}
%\begin{algorithmic}
%...
%\end{algorithmic}
%\end{algorithm}
%
%
%%% CHEMICAL FORMULAS AND REACTIONS
%
%%% For formulas embedded in the text, please use \chem{}
%
%%% The reaction environment creates labels including the letter R, i.e. (R1), (R2), etc.
%
%\begin{reaction}
%%% \rightarrow should be used for normal (one-way) chemical reactions
%%% \rightleftharpoons should be used for equilibria
%%% \leftrightarrow should be used for resonance structures
%\end{reaction}
%
%
%%% PHYSICAL UNITS
%%%
%%% Please use \unit{} and apply the exponential notation

\end{document}

% vim: set spell spelllang=en_uk tabstop=8 expandtab shiftwidth=2 softtabstop=2
